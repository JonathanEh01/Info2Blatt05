\documentclass[paper=a4, % Seitenformat
         fontsize=10pt,  % Schriftgröße
         oneside,        % einseitig
         headsepline,    % Trennlinie für die Kopfzeile
         notitlepage     % keine extra Titelseite
]{scrartcl}              % KOMA-Script Article
%------------------------------------------------------------------------

\usepackage[automark]{scrlayer-scrpage}  % Seiten-Stil für scrartcl
\usepackage[top=25mm]{geometry}  		% Oberer Rand 25mm Einrückung
\usepackage[utf8]{inputenc}              % Eingabekodierungen
\usepackage[T1]{fontenc}                 % Eingabekodierungen
\usepackage[english,ngerman]{babel}      % Mehrsprachenumgebung, Hauptsprache Deutsch
\usepackage{setspace}                    % Zeilenabstand
\usepackage{latexsym}                    % Latex-Symbole
\usepackage{amsfonts,amssymb,amstext}    % Mathematische Formeln
\usepackage{bbm}                         % bbm Schriftart
\usepackage{graphicx}                    % Abbildungen einbinden
\usepackage{listings}					%Programmcode einbingen
\usepackage{xcolor}						% Farben
\usepackage{changepage}
\usepackage{xparse}
\usepackage{hyperref}
\usepackage{cleveref}

\usepackage{tikz}
\usepackage{tikz-uml}

\ExplSyntaxOn
\NewDocumentCommand{\replace}{mmm}
{
	\marian_replace:nnn {#1} {#2} {#3}
}

\tl_new:N \l_marian_input_text_tl

\cs_new_protected:Npn \marian_replace:nnn #1 #2 #3
{
	\tl_set:Nn \l_marian_input_text_tl { #1 }
	\tl_replace_all:Nnn \l_marian_input_text_tl { #2 } { #3 }
	\tl_use:N \l_marian_input_text_tl
}
\ExplSyntaxOff


\pagestyle{scrheadings}					% Kopfzeilen nach scr-Standard		

% Definition von Befehlen, um die Vorlage nützlicher und verständlicher zu machen
\newcommand{\codeind}[1]{\begin{adjustwidth}{3.5mm}{}#1\end{adjustwidth}}
\usepackage{xparse}

\NewDocumentCommand{\includecode}{m o o}{%
  \codeind{\lstinputlisting[style=codestyle, language=Java, 
    firstline=\IfValueTF{#2}{#2}{1}, 
    lastline=\IfValueTF{#3}{#3}{2000}]{"src/#1"}}}

\newcommand{\includecodewithfilename}[1]{Datei: \texttt{\replace{#1}{_}{\_}}\vspace*{-1.5mm}\includecode{#1}}
\newcommand{\ownline}{\vspace{.7em}\hrule\vspace{.7em}} 
\newcommand{\aufgabe}[1]{\section*{Aufgabe #1}}

% Definition von Farben für die Codeblöcke
\definecolor{codegreen}{rgb}{0,0.6,0}
\definecolor{codegray}{rgb}{0.5,0.5,0.5}
\definecolor{codeblue}{rgb}{0.0, 0.0, 1.0}
\definecolor{bgcolour}{rgb}{0.97,0.97,0.97}
\definecolor{codered}{rgb}{0.7, 0.13, 0.13}

% Definition eines Designs für die Codeblöcke
\lstdefinestyle{codestyle}{
	backgroundcolor=\color{bgcolour},
	commentstyle=\color{codegreen},
	keywordstyle=\color{codeblue},
	numberstyle=\tiny\color{codegray},
	stringstyle=\color{codered},
	basicstyle=\ttfamily,
	numberstyle=\ttfamily\tiny\color{codegray},
	breakatwhitespace=false,         
	breaklines=true,                 
	captionpos=b,
	extendedchars=true                    
	keepspaces=true,                 
	numbers=left,                    
	numbersep=5pt,                  
	showspaces=false,                
	showstringspaces=false,
	showtabs=false,                  
	tabsize=4	
}

% Erlaubt es uns, in Codeblöcken Umlaute zu verwenden
\lstset{literate=%
	{Ö}{{\"O}}1
	{Ä}{{\"A}}1
	{Ü}{{\"U}}1
	{ß}{{\ss}}1
	{ü}{{\"u}}1
	{ä}{{\"a}}1
	{ö}{{\"o}}1
}



\parindent0em

\begin{document}

% Kopf des Dokuments
\includegraphics[width=0.90\textwidth]{images/logo.png} \\
\textbf{Übung zur Vorlesung Informatik 2} \hfill{SoSe 2025} \\  
Fakultät für Angewandte Informatik \\
Institut für Informatik \\
\textsc{Prof. B. Bauer, J. Linne, V. Le Claire, F. Stieler} \\
\mbox{} \\
{\large Übungsgruppe 99} % Nummer der Übungsgruppe einfügen
\ownline
\begin{center}
	{\LARGE \textbf{Abgabe des 4. Übungsblatts}} \\ % Nummer des Blatts einfügen
	\mbox{} \\
	{\large Jonathan Ehrenwirth, Michael Kraus, Florian Wissel} \\ % Namen der Teammitglieder einfügen
\end{center}
\ownline


%%%%%%%%%%%%%%%%%%%%%%%%%%%%%%%%%%%%%%%%%%%%%%%%%%%%%%%%%%%%%%%%%%%
%%%%%%%%%%%%%%%%%%% AB HIER BEARBEITEN %%%%%%%%%%%%%%%%%%%%%%%%%%%%
%%%%%%%%%%%%%%%%%%%%%%%%%%%%%%%%%%%%%%%%%%%%%%%%%%%%%%%%%%%%%%%%%%%
\newcommand{\teilaufgabe}[1]{\medskip\subsection*{Zu #1)}\medskip}

\aufgabe{13}

Das folgende Listing zeigt die Implementierung einer Programmklasse mit allen entsprechend der Teilaufgaben geforderten, überladenen Methoden.
\includecodewithfilename{Aufgabe13/Aufgabe13.java}

\aufgabe{14}

Das folgende Listing zeigt die Implementierung einer Programmklasse, die die Datei \texttt{numbers.txt} einliest und ein zweidimensionales Array daraus erstellt.
\includecodewithfilename{Aufgabe14/Aufgabe14.java}

\aufgabe{15}

\teilaufgabe{a}
\includecodewithfilename{Aufgabe15/Aufgabe15a.java}

\teilaufgabe{b}
\includecodewithfilename{Aufgabe15/Aufgabe15b.java}

\teilaufgabe{c}
\includecodewithfilename{Aufgabe15/Aufgabe15c.java}

\aufgabe{16}

\teilaufgabe{a}
\includecodewithfilename{Aufgabe16/Aufgabe16a.java}

\teilaufgabe{b}
\includecodewithfilename{Aufgabe16/Aufgabe16b.java}

\teilaufgabe{c}
\includecodewithfilename{Aufgabe16/Aufgabe16c.java}

\end{document}
